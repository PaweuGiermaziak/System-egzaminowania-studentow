\documentclass[a4paper]{article}
\usepackage{polski}
\usepackage[cp1250]{inputenc}
\usepackage{url}

\title{\bf{System egzaminowania studentów}}
\author{{\em Paweł Tomasiak, Kamil Orzechowski, Przemek Rysiewicz}}
\date{}

\begin{document}

\begin{titlepage}
\maketitle
\thispagestyle{empty}
\bigskip
\begin{center}
Zespołowe przedsięwzięcie inżynierskie\\[2mm]

Informatyka\\[2mm]

Rok. akad. 2017/2018, sem. I\\[2mm]

Prowadzący: dr hab. Marcin Mazur
\end{center}
\end{titlepage}

\tableofcontents
\thispagestyle{empty}

\newpage

\section{Opis projektu}

\subsection{Członkowie zespołu}

\begin{enumerate}
\item Paweł Tomasiak (kierownik projektu)
\item Przemysław Rysiewicz
\item Kamil Orzechowski
\end{enumerate}

\subsection{Cel projektu (produkt)}

Celem projektu jest stworzenie platformy ułatwiającej osobie prowadzącej zadania dydaktyczne egzaminowanie studentów. Platforma ma w założeniu być substytutem testu wiedzy z pełną automatyzacją obsługi oceniania.

\subsection{Potencjalny odbiorca produktu (klient)}

Odbiorcą produktu jest osoba prowadzaca zadania dydaktyczne wymagajace oceniania wiedzy osób nauczanch. Do tej grupy zalicza sie nauczycieli i wykładowców.

\subsection{Metodyka}

Projekt będzie realizowany przy użyciu (zaadaptowanej do istniejących warunków) metodyki {\em Scrum}. 

\section{Wymagania użytkownika}
\subsection{User story 1}

Jako student chcę mieć przyjazny interfejs graficzny, w celu przyjaznego i intuicyjnego korzystania ze skryptu oceniającego wiedzę

\subsection{User story 2}

Jako student chcę mieć możliwość wyboru interesujecego mnie kolokwium w celu rozpoczecia jego uzupełniania

\subsection{User story 3}

Jako student chcę widzieć wynik jaki osiągnąłem podczas egzaminu

\subsection{User story 4}

Jako student chcę mieć możliwość logowania sie na istniejące konta w bazie

\subsection{User story 5}

Jako student chcę mieć możliwość wybrania konkretnego zespołu (roku akademickiego)

\subsection{User story 6}

Jako prowadzacy zajęcia chcę aby skrypt przesłał dane o wyniku na serwer w celu możliwości ich odczytania

\subsection{User story 7}

Jako prowadzący zajęcia chcę aby skrypt generował pyrania z puli

\subsection{User story 8}

Jako prowadzący zajęcia chcę aby wybór poprawnej odpowiedzi odbywał się poprzez zaznaczenie jednej z 4 możliwych

\subsection{User story 9}

Jako prowadzacy zajęcia chcę żeby ilość czasu na wypenienie testu była ogarniczna.

\subsection{User story 10}

Jako prowadzący zajęcia chcę aby po zakończeniu dostępnego czasu skrypt wyświetlał uzyskany wynik

\subsection{User story 11}

Jako prowadzący chcę aby skrypt zapisywał wynik otrzymany przez studenta na serwerze

\subsection{User story 12}

Jako wykładowca wymagam dostepu do danych przechowywanych na serwerze bazy w sposób przystępny i zorganizowany

\subsection{User story 13}

Jako prowadzacy zajecia mam mieć mozliwosc dodawania pytan do bazy w celu swobodnego dostosowania pytań egzaminiu do aktualnych potrzeb

\subsection{User story 14}

Jako wykładowca wymagam możliwości dodawania kont sudentów

\subsection{User story 15}

Jako prowadzący chcę mieć możliość wyświetlenia tabeli przedstawiającej osiągniecia poszczególnych studentów

\section{Harmonogram}

\subsection{Rejestr zadań (Product Backlog)}

\begin{itemize}
\item Data rozpoczęcia: <<data>>.
\item  Data zakończenia: <<data>>.
\end{itemize}

\subsection{Sprint 1}

\begin{itemize}
\item Data rozpoczęcia: <<2017-10-18>>.
\item Data zakończenia: <<data>>.
\item Scrum Master: <<Paweł Tomasiak>>.
\item Product Owner: <<Paweł Tomasiak>>.
\item Development Team: <<Paweł Tomasiak, Przemysław Rysiewicz, Kamil Orzechowski>>.
\end{itemize}

\subsection{Sprint 2}

\begin{itemize}
\item Data rozpoczęcia: <<data>>.
\item  Data zakończenia: <<data>>.
\item Scrum Master: <<Paweł Tomasiak>>.
\item Product Owner: <<Paweł Tomasiak>>.
\item Development Team: <<Paweł Tomasiak, Przemysław Rysiewicz, Kamil Orzechowski>>.
\end{itemize}

\subsection*{<<Tutaj dodawać kolejne Sprint'y>>}

\section{Product Backlog}

\subsection{Backlog Item 1}
\paragraph{Tytuł zadania.} <<Prototyp egzaminatora>>.
\paragraph{Opis zadania.} <<Stworzenie działającego prototypu strony będącego w stanie egzaminować uczniów>>.
\paragraph{Priorytet.} <<Priorytet>>.
\paragraph{Definition of Done.} <<Określić (w języku zrozumiałym dla wszystkich członków zespołu), co oznacza ukończenie danego zadania>>.

\subsection{Backlog Item 2}
\paragraph{Tytuł zadania.} <<Tytuł>>.
\paragraph{Opis zadania.} <<Opis>>.
\paragraph{Priorytet.} <<Priorytet>>.
\paragraph{Definition of Done.} <<Określić (w języku zrozumiałym dla wszystkich członków zespołu), co oznacza ukończenie danego zadania>>.

\subsection*{<<Tutaj dodawać kolejne zadania>>}

\section{Sprint 1}
\subsection{Cel} <<Określić, w jakim celu tworzony jest przyrost produktu>>.
\subsection{Sprint Planning/Backlog}

\paragraph{Tytuł zadania.} <<Tytuł>>.
\begin{itemize}
\item Estymata: <<szacowana czasochłonność (w ,,koszulkach'')>>.
\end{itemize}

\paragraph{Tytuł zadania.} <<Tytuł>>.
\begin{itemize}
\item Estymata: <<szacowana czasochłonność (w ,,koszulkach'')>>.
\end{itemize}

\paragraph{<<Tutaj dodawać kolejne zadania>>}

\subsection{Realizacja}

\paragraph{Tytuł zadania.} <<Tytuł>>.
\subparagraph{Wykonawca.} <<Wykonawca>>.
\subparagraph{Realizacja.} <<Sprawozdanie z realizacji zadania (w tym ocena zgodności z estymatą). Kod programu (środowisko \texttt{verbatim}): \begin{verbatim}
for (i=1; i<10; i++)
...
\end{verbatim}>>.

\paragraph{Tytuł zadania.} <<Tytuł>>.
\subparagraph{Wykonawca.} <<Wykonawca>>.
\subparagraph{Realizacja.} <<Sprawozdanie z realizacji zadania (w tym ocena zgodności z estymatą). Kod programu (środowisko \texttt{verbatim}): \begin{verbatim}
for (i=1; i<10; i++)
...
\end{verbatim}>>.

\paragraph{<<Tutaj dodawać kolejne zadania>>}


\subsection{Sprint Review/Demo}
<<Sprawozdanie z przeglądu Sprint'u -- czy założony cel (przyrost) został osiągnięty oraz czy wszystkie zaplanowane Backlog Item'y zostały zrealizowane? Demostracja przyrostu produktu>>.

\section{Sprint 2}

\subsection{Cel} <<Określić, w jakim celu tworzony jest przyrost produktu>>.

\subsection{Sprint Planning/Backlog}

\paragraph{Tytuł zadania.} <<Tytuł>>.
\begin{itemize}
\item Estymata: <<szacowana czasochłonność (w ,,koszulkach'')>>.
\end{itemize}

\paragraph{Tytuł zadania.} <<Tytuł>>.
\begin{itemize}
\item Estymata: <<szacowana czasochłonność (w ,,koszulkach'')>>.
\end{itemize}

\paragraph{<<Tutaj dodawać kolejne zadania>>}

\subsection{Realizacja}

\paragraph{Tytuł zadania.} <<Tytuł>>.
\subparagraph{Wykonawca.} <<Wykonawca>>.
\subparagraph{Realizacja.} <<Sprawozdanie z realizacji zadania (w tym ocena zgodności z estymatą). Kod programu (środowisko \texttt{verbatim}): \begin{verbatim}
for (i=1; i<10; i++)
...
\end{verbatim}>>.

\paragraph{Tytuł zadania.} <<Tytuł>>.
\subparagraph{Wykonawca.} <<Wykonawca>>.
\subparagraph{Realizacja.} <<Sprawozdanie z realizacji zadania (w tym ocena zgodności z estymatą). Kod programu (środowisko \texttt{verbatim}): \begin{verbatim}
for (i=1; i<10; i++)
...
\end{verbatim}>>.

\paragraph{<<Tutaj dodawać kolejne zadania>>}


\subsection{Sprint Review/Demo}
<<Sprawozdanie z przeglądu Sprint'u -- czy założony cel (przyrost) został osiągnięty oraz czy wszystkie zaplanowane Backlog Item'y zostały zrealizowane? Demostracja przyrostu produktu>>.

\section*{<<Tutaj dodawać kolejne Sprint'y>>}


\begin{thebibliography}{9}

\bibitem{Cov} S. R. Covey, {\em 7 nawyków skutecznego działania}, Rebis, Poznań, 2007.

\bibitem{Oet} Tobias Oetiker i wsp., Nie za krótkie wprowadzenie do systemu \LaTeX  \ $2_\varepsilon$, \url{ftp://ftp.gust.org.pl/TeX/info/lshort/polish/lshort2e.pdf}

\bibitem{SchSut} K. Schwaber, J. Sutherland, {\em Scrum Guide}, \url{http://www.scrumguides.org/}, 2016.

\bibitem{apr} \url{https://agilepainrelief.com/notesfromatooluser/tag/scrum-by-example}

\bibitem{us} \url{https://www.tutorialspoint.com/scrum/scrum_user_stories.htm}

\end{thebibliography}

\end{document}
