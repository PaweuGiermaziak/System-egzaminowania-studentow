\documentclass[a4paper]{article}
\usepackage{polski}
\usepackage[cp1250]{inputenc}
\usepackage{url}

\title{\bf{System egzaminowania studentów}}
\author{{\em Paweł Tomasiak, Kamil Orzechowski, Przemek Rysiewicz}}
\date{}

\begin{document}

\begin{titlepage}
\maketitle
\thispagestyle{empty}
\bigskip
\begin{center}
Zespołowe przedsięwzięcie inżynierskie\\[2mm]

Informatyka\\[2mm]

Rok. akad. 2017/2018, sem. I\\[2mm]

Prowadzący: dr hab. Marcin Mazur
\end{center}
\end{titlepage}

\tableofcontents
\thispagestyle{empty}

\newpage

\section{Opis projektu}

\subsection{Członkowie zespołu}

\begin{enumerate}
\item <<Paweł Tomasiak>> (kierownik projektu).
\item <<Przemysław Rysiewicz>>.
\item <<Kamil Orzechowski>>.
\end{enumerate}

\subsection{Cel projektu (produkt)}

Celem projektu jest stworzenie platformy ułatwiającej osobie prowadzącej zadania dydaktyczne egzaminowanie studentów. Platforma ma w założeniu być substytutem testu wiedzy z pełną automatyzacją obsługi oceniania.

\subsection{Potencjalny odbiorca produktu (klient)}

Odbiorcą produktu jest osoba prowadzaca zadania dydaktyczne wymagajace oceniania wiedzy osób nauczanch. Do tej grupy zalicza sie nauczycieli i wykładowców.

\subsection{Metodyka}

Projekt będzie realizowany przy użyciu (zaadaptowanej do istniejących warunków) metodyki {\em Scrum}. 

\section{Wymagania użytkownika}
						<<Przedstawić listę wymagań użytkownika w postaci ,,historyjek'' (User stories). Każda historyjka powinna opisywać jedną cechę systemu. Struktura: As a [type of user], I want [to perform some task] so that I can [achieve some goal/benefit/value] (zob. np. \cite{us}).>>

\subsection{User story 1}
1. 
Jako student chcę mieć przyjazny interfejs graficzny, w celu przyjaznego i intuicyjnego korzystania ze skryptu oceniającego wiedzę.

Projekt graficzny ma zawierać:
- projekt strony logowania
- projekt strony wyboru kolokwium/egzaminu
- projekt strony panel wyboru odpowiedzi
- projekt strony informacji o wyniku
- projekt strony panelu adminstartora.


\subsection{User story 2}
2.
Jako student chcę mieć możliwość zalogowania sie na moje istniejące konto stworzone przez prowadzacego zajęcia i mieć możliwosc wyboru interesujecego mnie kolokwium w celu rozpoczecia jego uzupełniania.

Projekt strony internetowej na zawierać:
- możliwosc logowania sie na istejące konta w bazie
- można wybrac dowolne przeznaczone dla danego sudenta kolokium 
- można rozpocząc wypełnianie testu


\subsection*{User story 3}
3.
Jako student chcę mieć możliwość wypełnienia kolokwium na stornie internetowej w celu zalicenia testu. Jako prowadzacy zajecia chce żeby ilość czasu na wypenienie testu była ogarniczna. Jako student po za konczeniu testu oczekuję na otrzymanie wyniku w postaci ułamka dziesiętnego w celu uzyskania informacji o wyniku testu. Jako prowadzacy wymagam aby skrypt przesłał dane o wyniku na serwer w celu mozliwosci ich odczytania.

Projekt strony internetowej na zawierać:
- możliwość wypełnienia wygenerowanych przez skrypt pytań
- wybór poprawnej odpowiedzi odbywa sie poprzez zaznaczenie jednej z 4 możliwych
- skrypt obsługuje przewijanie kolejnych pytań
- po zakończeniu egzaminu skrypt wyświetla uzyskany wynik w formie x.x/10
- po zakończeniu dostępnego czasu skrypt wyświetla uzyskany wynik w formie x.x/10        pokazywanie wyników na ektanie
- w momencie wyświetlania skrypt zapisuje wynik egzaminu na serwerze.


///pozordzielaic historyjki
\subsection*{User story 4}
<<4.
Jako wykładowca wymagam dostepu do danych przechowywanych na serwerze bazy w sposób przystępny i zorganizowany, oparty na koncepcji tabelki w celu przystepnego ich odczytania. Jako wykładowca wymagam mozliwosci dodawania kont sudentów w celu dodawania kont studentów. Jako prowadzacy zajecia mam mieć mozliwosc dodawania pytan do bazy w celu swobodnego dostosowania putan egzaminiu do akutalnych potrzeb. 

Projekt strony internetowej na zawierać:
- możliwość zalogowania sie jako administrator
- możliwość wybrania konkretnego zespołu (roku akademickiego)
- mozliosc wyswietlanie tabeli przedstawiajacej osiagniecia poszczególnych studentów>>.

\section{Harmonogram}

\subsection{Rejestr zadań (Product Backlog)}

\begin{itemize}
\item Data rozpoczęcia: <<data>>.
\item  Data zakończenia: <<data>>.
\end{itemize}

\subsection{Sprint 1}

\begin{itemize}
\item Data rozpoczęcia: <<2017-10-18>>.
\item Data zakończenia: <<data>>.
\item Scrum Master: <<Paweł Tomasiak>>.
\item Product Owner: <<Paweł Tomasiak>>.
\item Development Team: <<Paweł Tomasiak, Przemysław Rysiewicz, Kamil Orzechowski>>.
\end{itemize}

\subsection{Sprint 2}

\begin{itemize}
\item Data rozpoczęcia: <<data>>.
\item  Data zakończenia: <<data>>.
\item Scrum Master: <<Paweł Tomasiak>>.
\item Product Owner: <<Paweł Tomasiak>>.
\item Development Team: <<Paweł Tomasiak, Przemysław Rysiewicz, Kamil Orzechowski>>.
\end{itemize}

\subsection*{<<Tutaj dodawać kolejne Sprint'y>>}

\section{Product Backlog}

\subsection{Backlog Item 1}
\paragraph{Tytuł zadania.} <<Prototyp egzaminatora>>.
\paragraph{Opis zadania.} <<Stworzenie działającego prototypu strony będącego w stanie egzaminować uczniów>>.
\paragraph{Priorytet.} <<Priorytet>>.
\paragraph{Definition of Done.} <<Określić (w języku zrozumiałym dla wszystkich członków zespołu), co oznacza ukończenie danego zadania>>.

\subsection{Backlog Item 2}
\paragraph{Tytuł zadania.} <<Tytuł>>.
\paragraph{Opis zadania.} <<Opis>>.
\paragraph{Priorytet.} <<Priorytet>>.
\paragraph{Definition of Done.} <<Określić (w języku zrozumiałym dla wszystkich członków zespołu), co oznacza ukończenie danego zadania>>.

\subsection*{<<Tutaj dodawać kolejne zadania>>}

\section{Sprint 1}
\subsection{Cel} <<Określić, w jakim celu tworzony jest przyrost produktu>>.
\subsection{Sprint Planning/Backlog}

\paragraph{Tytuł zadania.} <<Tytuł>>.
\begin{itemize}
\item Estymata: <<szacowana czasochłonność (w ,,koszulkach'')>>.
\end{itemize}

\paragraph{Tytuł zadania.} <<Tytuł>>.
\begin{itemize}
\item Estymata: <<szacowana czasochłonność (w ,,koszulkach'')>>.
\end{itemize}

\paragraph{<<Tutaj dodawać kolejne zadania>>}

\subsection{Realizacja}

\paragraph{Tytuł zadania.} <<Tytuł>>.
\subparagraph{Wykonawca.} <<Wykonawca>>.
\subparagraph{Realizacja.} <<Sprawozdanie z realizacji zadania (w tym ocena zgodności z estymatą). Kod programu (środowisko \texttt{verbatim}): \begin{verbatim}
for (i=1; i<10; i++)
...
\end{verbatim}>>.

\paragraph{Tytuł zadania.} <<Tytuł>>.
\subparagraph{Wykonawca.} <<Wykonawca>>.
\subparagraph{Realizacja.} <<Sprawozdanie z realizacji zadania (w tym ocena zgodności z estymatą). Kod programu (środowisko \texttt{verbatim}): \begin{verbatim}
for (i=1; i<10; i++)
...
\end{verbatim}>>.

\paragraph{<<Tutaj dodawać kolejne zadania>>}


\subsection{Sprint Review/Demo}
<<Sprawozdanie z przeglądu Sprint'u -- czy założony cel (przyrost) został osiągnięty oraz czy wszystkie zaplanowane Backlog Item'y zostały zrealizowane? Demostracja przyrostu produktu>>.

\section{Sprint 2}

\subsection{Cel} <<Określić, w jakim celu tworzony jest przyrost produktu>>.

\subsection{Sprint Planning/Backlog}

\paragraph{Tytuł zadania.} <<Tytuł>>.
\begin{itemize}
\item Estymata: <<szacowana czasochłonność (w ,,koszulkach'')>>.
\end{itemize}

\paragraph{Tytuł zadania.} <<Tytuł>>.
\begin{itemize}
\item Estymata: <<szacowana czasochłonność (w ,,koszulkach'')>>.
\end{itemize}

\paragraph{<<Tutaj dodawać kolejne zadania>>}

\subsection{Realizacja}

\paragraph{Tytuł zadania.} <<Tytuł>>.
\subparagraph{Wykonawca.} <<Wykonawca>>.
\subparagraph{Realizacja.} <<Sprawozdanie z realizacji zadania (w tym ocena zgodności z estymatą). Kod programu (środowisko \texttt{verbatim}): \begin{verbatim}
for (i=1; i<10; i++)
...
\end{verbatim}>>.

\paragraph{Tytuł zadania.} <<Tytuł>>.
\subparagraph{Wykonawca.} <<Wykonawca>>.
\subparagraph{Realizacja.} <<Sprawozdanie z realizacji zadania (w tym ocena zgodności z estymatą). Kod programu (środowisko \texttt{verbatim}): \begin{verbatim}
for (i=1; i<10; i++)
...
\end{verbatim}>>.

\paragraph{<<Tutaj dodawać kolejne zadania>>}


\subsection{Sprint Review/Demo}
<<Sprawozdanie z przeglądu Sprint'u -- czy założony cel (przyrost) został osiągnięty oraz czy wszystkie zaplanowane Backlog Item'y zostały zrealizowane? Demostracja przyrostu produktu>>.

\section*{<<Tutaj dodawać kolejne Sprint'y>>}


\begin{thebibliography}{9}

\bibitem{Cov} S. R. Covey, {\em 7 nawyków skutecznego działania}, Rebis, Poznań, 2007.

\bibitem{Oet} Tobias Oetiker i wsp., Nie za krótkie wprowadzenie do systemu \LaTeX  \ $2_\varepsilon$, \url{ftp://ftp.gust.org.pl/TeX/info/lshort/polish/lshort2e.pdf}

\bibitem{SchSut} K. Schwaber, J. Sutherland, {\em Scrum Guide}, \url{http://www.scrumguides.org/}, 2016.

\bibitem{apr} \url{https://agilepainrelief.com/notesfromatooluser/tag/scrum-by-example}

\bibitem{us} \url{https://www.tutorialspoint.com/scrum/scrum_user_stories.htm}

\end{thebibliography}

\end{document}
