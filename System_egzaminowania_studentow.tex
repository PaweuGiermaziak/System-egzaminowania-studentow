\documentclass[a4paper]{article}
\usepackage{polski}
\usepackage[cp1250]{inputenc}
\usepackage{url}

\title{\bf{System egzaminowania studentów}}
\author{{\em Paweł Tomasiak, Kamil Orzechowski, Przemek Rysiewicz}}
\date{}

\begin{document}

\begin{titlepage}
\maketitle
\thispagestyle{empty}
\bigskip
\begin{center}
Zespołowe przedsięwzięcie inżynierskie\\[2mm]

Informatyka\\[2mm]

Rok. akad. 2017/2018, sem. I\\[2mm]

Prowadzący: dr hab. Marcin Mazur
\end{center}
\end{titlepage}

\tableofcontents
\thispagestyle{empty}

\newpage

\section{Opis projektu}

\subsection{Członkowie zespołu}

\begin{enumerate}
\item Paweł Tomasiak (kierownik projektu)
\item Przemysław Rysiewicz
\item Kamil Orzechowski
\end{enumerate}

\subsection{Cel projektu (produkt)}

Celem projektu jest stworzenie platformy ułatwiającej osobie prowadzącej zadania dydaktyczne egzaminowanie studentów. Platforma ma w założeniu być substytutem testu wiedzy z pełną automatyzacją obsługi oceniania.

\subsection{Potencjalny odbiorca produktu (klient)}

Odbiorcą produktu jest osoba prowadzaca zadania dydaktyczne wymagajace oceniania wiedzy osób nauczanch. Do tej grupy zalicza sie nauczycieli i wykładowców.

\subsection{Metodyka}

Projekt będzie realizowany przy użyciu (zaadaptowanej do istniejących warunków) metodyki {\em Scrum}. 

\section{Wymagania użytkownika}
\subsection{User story 1}

Jako użytkownik chcę aby interfejs graficzny był przejrzysty tak abym mógł intuicyjnie poruszać się po stronie

\subsection{User story 2}

Jako użytkownik chcę mieć łatwy dostęp do interesującego mnie kolokwium żeby od razu móc przystąpić do testu

\subsection{User story 3}

Jako użytkownik chcę widzieć wynik jaki osiągnąłem podczas egzaminu aby móc śledzić swoje postępy

\subsection{User story 4}

Jako użytkownik chcę mieć możliwość logowania sie na istniejące konta w bazie, żeby w każdej chwili mieć dostęp do testów

\subsection{User story 5}

Jako użytkownik chcę mieć możliwość wybrania konkretnego zespołu (roku akademickiego), żeby widzieć tylko dotyczące mnie egzaminy

\subsection{User story 6}

Jako administrator chcę aby skrypt przesłał dane o rezultacie testu na serwer tak abym mógł obejrzeć wyniki każdego użytkownika

\subsection{User story 7}

Jako administrator chcę aby skrypt generował pyrania z puli, po to żeby umożliwić dobór różnych pytań dla każdego studenta

\subsection{User story 8}

Jako administrator chcę aby wybór poprawnej odpowiedzi mógł odbywać się poprzez zaznaczenie więcej niż jednej z kilku możliwych, abym miał więcej możliwości w tworzeniu pytań

\subsection{User story 9}

Jako administrator chcę żeby ilość czasu na wypenienie testu była ograniczona, żeby egzamin nie trwał dłużej jest niż to konieczne

\subsection{User story 10}

Jako administrator chcę aby po zakończeniu dostępnego czasu skrypt wyświetlał uzyskany wynik, żeby każdy dysponował taką samą ilością czasu

\subsection{User story 11}

Jako administrator chcę aby skrypt zapisywał wynik otrzymany przez studenta na serwerze, tak abym miał do niego później dostęp

\subsection{User story 12}

Jako administrator wymagam dostepu do danych przechowywanych na serwerze bazy w sposób przystępny i zorganizowany, tak abym mógł łatwo przeglądać poszczególne wyniki

\subsection{User story 13}

Jako administrator chcę mieć możliwość dodawania pytań do bazy w celu swobodnego dostosowywania zadań egzaminu do aktualnych potrzeb

\subsection{User story 14}

Jako wykładowca wymagam możliwości dodawania kont sudentów, tak abym mógł bezpiecznie zarządzać ich ilością

\section{Harmonogram}

\subsection{Rejestr zadań (Product Backlog)}

\begin{itemize}
\item Data rozpoczęcia: 2017-10-18.
\item  Data zakończenia: 2017-11-15.
\end{itemize}


\subsection{Sprint 1}

\begin{itemize}
\item Data rozpoczęcia: 2017-11-15.
\item  Data zakończenia: 2017-11-22.
\item Scrum Master: Kamil Orzechowski
\item Product Owner: Przemysław Rysiewicz
\item Development Team: Paweł Tomasiak, Przemysław Rysiewicz, Kamil Orzechowski.
\end{itemize}

\subsection{Sprint 2}

\begin{itemize}
\item Data rozpoczęcia: 2017-11-22.
\item  Data zakończenia: 2017-12-13.
\item Scrum Master: Kamil Orzechowski
\item Product Owner: Przemysław Rysiewicz
\item Development Team: Paweł Tomasiak, Przemysław Rysiewicz, Kamil Orzechowski.
\end{itemize}

\subsection{Sprint 3}

\begin{itemize}
\item Data rozpoczęcia: 2017-12-13.
\item  Data zakończenia: 2018-01-10.
\item Scrum Master: Paweł Tomasiak
\item Product Owner: Kamil Orzechowski
\item Development Team: Paweł Tomasiak, Przemysław Rysiewicz, Kamil Orzechowski.
\end{itemize}

\subsection{Sprint 4}

\begin{itemize}
\item Data rozpoczęcia: 2018-01-10.
\item  Data zakończenia: 2018-01-24.
\item Scrum Master: Przemysław Rysiewicz
\item Product Owner: Paweł Tomasiak
\item Development Team: Kamil Orzechowski, Przemysław Rysiewicz
\end{itemize}


\section{Product Backlog}

\subsection{Backlog Item 1}
\paragraph{Tytuł zadania.} Prototyp egzaminatora.
\paragraph{Opis zadania.} Stworzenie działającego prototypu strony będącego w stanie egzaminować uczniów.
\paragraph{Priorytet.} Wysoki.
\paragraph{Definition of Done.} Prototyp zawiera podgląd strony, która posłuży do przeprowadzania egzaminu, podstrony: logowanie, lista egzaminów, zakończenie egzaminu oraz administrator.

\subsection{Backlog Item 2}
\paragraph{Tytuł zadania.} Baza danych.
\paragraph{Opis zadania.} Stworzenie struktur bazodanowych w środowisku MySQL.
\paragraph{Priorytet.} Wysoki.
\paragraph{Definition of Done.} Stworzenie bazy danych posiadającej wymagane tabele. Wstępna wizja projektu zakłada utworzenie tabel takich jak: "użytkownicy" zawierająca informacje o utworzonych kontach użytkowników (Imię, nazwisko oraz login i  hasło); tabele "działy" oraz "kolokwia", przechowujące nazwy kolokwiów nazwy kolokwiów oraz nazwy działów do których kolokwia są przypisane; najobszerniejszej tabeli "pytania" zawierającej treści pytań egzaminów oraz odpowiedzi do nich; tabela "sesja" zawierająca informacje o przeprowadzonych egzaminach przez danego użytkownika.  

\subsection{Backlog Item 3}
\paragraph{Tytuł zadania.} Rejestracja.
\paragraph{Opis zadania.} Możliwość utowrzenia nowego konta.
\paragraph{Priorytet.} Średni.
\paragraph{Definition of Done.} Stworzenie formularza umożliwiającego założenie nowego konta użytkownika. Pomyślna rejestracja utworzy nowy rekord w bazie danych.

\subsection{Backlog Item 4}
\paragraph{Tytuł zadania.} Weryfikacja.
\paragraph{Opis zadania.} Zweryfikowanie wprowadzonych danych podczas logowania.
\paragraph{Priorytet.} Średni.
\paragraph{Definition of Done.} Sprawdzenie, czy podane zostały wszystkie wymagane dane, czy posiadają odpowiedni format oraz czy liczba wprowadzonych znaków mieści się w ustalonym przedziale. Jeśli nie został spełniony któryś z warunków zostanie wyświetlony komunikat.

\subsection{Backlog Item 5}
\paragraph{Tytuł zadania.} Logowanie.
\paragraph{Opis zadania.} Umożliwienie zalogowania na istniejące konto.
\paragraph{Priorytet.} Wysoki.
\paragraph{Definition of Done.} Formularz logowania umożliwiający rozpoczęcie nowej sesji oraz dostęp do przeznaczonych dla niego egzaminów. Zalogowanie nastąpi jeśli wprowadzone w formularzu dane (login,hasło) znajdują się już w istniejącej bazie.

\subsection{Backlog Item 6}
\paragraph{Tytuł zadania.} Lista egzaminów.
\paragraph{Opis zadania.} Panel boczny zawierający listę egzaminów.
\paragraph{Priorytet.} Wysoki.
\paragraph{Definition of Done.} Panel znajdujący się po lewej stronie zawierający rozwijaną listę aktualnie dostępnych egzaminów.

\subsection{Backlog Item 7}
\paragraph{Tytuł zadania.} Panel konta użytkownika.
\paragraph{Opis zadania.} Przegląd danych konta użytkownika oraz wyniki egzaminów .
\paragraph{Priorytet.} Śrerdni.
\paragraph{Definition of Done.} Użytkownik ma możliwość przeglądania danych wprowadzonych podczas rejestracji oraz zmiany niektórych z nich(adres email, hasło) , może przeglądać swoje wyniki z egzaminów .

\subsection{Backlog Item 8}
\paragraph{Tytuł zadania.} Przydzielanie egzaminów.
\paragraph{Opis zadania.} Dobranie egzaminów przez administratora.
\paragraph{Priorytet.} Niski
\paragraph{Definition of Done.} Użytkownik z prawami administratora ma możliwość dodawania nowych pytań do utworzonych wcześniej działów, do których dostęp ma później użytkownik - student.

\subsection{Backlog Item 9}
\paragraph{Tytuł zadania.} Zarządanie studentami.
\paragraph{Opis zadania.} Przydzielanie studentów do wybranego roku, usuwanie.
\paragraph{Priorytet.} Niski
\paragraph{Definition of Done.} Poprzez zarządzanie studentami rozumiemy możliwość wyświetlania wyników osiągniętych przez studentów podczas egzaminów.

\subsection{Backlog Item 10}
\paragraph{Tytuł zadania.} Serwer
\paragraph{Opis zadania.} Utworzenie serwera, na którym działał będzie egzaminator
\paragraph{Priorytet.} Wysoki
\paragraph{Definition of Done.} Skrypt przełącza użytkownika na stronę wyboru egzaminów po wpisaniu poprawnych danych logowania. 

\subsection*{Tutaj dodawać kolejne zadania}

\section{Sprint 1} 
\subsection{Cel} Przedstawienie projektu graficznego w wersji HTML
\subsection{Sprint Planning/Backlog}

\paragraph{Tytuł zadania.} Prototyp egzaminatora
\begin{itemize}
\item Estymata: M
\end{itemize}

\subsection{Realizacja}

\paragraph{Tytuł zadania.} Prototyp egzaminatora
\subparagraph{Wykonawca.} Paweł Tomasiak, Kamil Orzechowski
\subparagraph{Realizacja.} Kamil Orzechowski - Wykonanie projektu graficznego w formie jpg.
Paweł Tomasiak - Została utworzona strona logowania, która wymaga podania loginu i hasła; strona zawierająca rozwijaną listę egzaminów, z możliwością rozpoczęcia wybranego; stronę przedstawiającą wygląd rozpoczętego egzaminu, zawierająca licznik czasowy, panel wyboru pytań i okno z wybranym pytaniem wraz z odpowiedziami; strona wyświetlana po zakończeniu egzaminu zawierająca wynik; stron zawierająca panel administratora przedstawiający wyniki studentów.

\subsection{Sprint Review/Demo}
Założony cel sprintu został spełniony. Wszystkie założenia zostały osiągnięte w ustalonym czasie. Zrealizowany został prototyp zawierający strony:
\begin{itemize}
\item strona logowania
\item lista egzaminów
\item podgląd rozpoczętego egzaminu
\item wynik egzaminu 
\item panel administratora
\end{itemize}

Demonstracja została przeprowadzona dnia: 2017-11-22

\section{Sprint 2}

\subsection{Cel} Serwer, logowanie, weryfikacja, baza danych

\subsection{Sprint Planning/Backlog}

\paragraph{Tytuł zadania.} Serwer
\begin{itemize}
\item Estymata: L
\end{itemize}

\paragraph{Tytuł zadania.} Logowanie
\begin{itemize}
\item Estymata: L
\end{itemize}

\paragraph{Tytuł zadania.} Weryfikacja
\begin{itemize}
\item Estymata: M
\end{itemize}

\paragraph{Tytuł zadania.} Baza danych
\begin{itemize}
\item Estymata: L
\end{itemize}

\subsection{Realizacja}

\paragraph{Tytuł zadania.} Serwer
\subparagraph{Wykonawca.} Kamil Orzechowski
\subparagraph{Realizacja.} Został założony lokalny serwer, wraz z usługami php i sql. Serwer korzysta z programu WAMPserver. Realizacja tego kroku wymagała tylko instalacji oprogramowania oraz podstawowej konfiguracji (tj. umieszczenie plików projektu w odpowiednim folderze). Obsługa lokalnego serwera jest prosta, więc przygotowanie serwera do działania wraz z instalacją zajęła kilkanaście minut.  

\paragraph{Tytuł zadania.} Logowanie
\subparagraph{Wykonawca.} Paweł Tomasiak
\subparagraph{Realizacja.} Utworzony skrypt działa na bazie utworzonego w poprzednim sprincie formularza logowania. Po wpisaniu loginu oraz hasła za pomocą metody POST zostaje sprawdzona zgodność wpisanego loginu i hasła z informacjami w tabeli 'użytkownicy'. Logowanie obejmuje również działający przycisk wylogowania, który po użyciu uruchamia ponownie formularz logowania, uniemożliwiając dostęp do treści dla zalogowanych. Dzięki dobrej znajomości języka PHP przez naszych deweloperów skrypt został wykonany w trzy dni po ok. 1-1.5h pracy dziennie .

\paragraph{Tytuł zadania.} Weryfikacja
\subparagraph{Wykonawca.} Paweł Tomasiak
\subparagraph{Realizacja.} Skrypt na serwerze weryfikuje podane dane porównując je z wpisami z bazy sql. Po wprowadzeniu błędnych danych formularz logowania się resetuje oraz wyświetlony zostaje komunikat o błędnym loginie lub haśle. 


\paragraph{Tytuł zadania.} Baza Danych
\subparagraph{Wykonawca.} Przemysław Rysiewicz
\subparagraph{Realizacja.} Stworzona została baza danych posiadająca kilka tabel:  "użytkownicy" zawierająca informacje o utworzonych kontach użytkowników (Imię, nazwisko oraz login i  hasło); tabele "działy" oraz "kolokwia", przechowujące nazwy kolokwiów nazwy kolokwiów oraz nazwy działów do których kolokwia są przypisane; najobszerniejszej tabeli "pytania" zawierającej treści pytań egzaminów oraz odpowiedzi do nich; tabela "sesja" zawierająca informacje o przeprowadzonych egzaminach przez danego użytkownika. 
W każdej z tabel znajdują się jakieś pozycje, które na chwilę obecną wprowadzone są "na sztywno" w środowisku sql przy użyciu polecenia \emph{INSERT INTO TABLE}. Baza mimo iż nie posiada skomplikowanej struktury i samo tworzenie jej nie było zbyt problematyczne wymagała przemyślenia przez co prace nad nią potrwały łącznie ok 3-4 dni po 1-2h dziennie. 


	\vspace{0.5cm}
	Tabela \textbf{uzytkownicy}:
	\begin{itemize}
		\item \textbf{id} $\to$ typu - \verb|INT(11)|, \verb|AUTO_INCREMENT|],
		\item \textbf{user} $\to$ typu - \verb|VARCHAR(25)|, \verb|NOT NULL|,
		\item \textbf{password} $\to$ typu - \verb|INT(25)|, \verb|DEFAULT NULL| 
		\item \textbf{name} $\to$ typu - \verb|VARCHAR(35)|, \verb|NOT NULL| 
	\end{itemize}
	
		\vspace{0.5cm}
		Tabela \textbf{sessions}:
		\begin{itemize}
			\item \textbf{id} $\to$ typu - \verb|INT(11)|, \verb|AUTO_INCREMENT|,
			\item \textbf{idUser} $\to$ typu - \verb|INT(11)|, \verb|NOT NULL|,
			\item \textbf{nazwaSesji} $\to$ typu - \verb|INT(11)|, \verb|NOT NULL| 
			\item \textbf{idDzialu} $\to$ typu - \verb|INT(11)|, \verb|NOT NULL| 
			\item \textbf{idKolokwium} $\to$ typu - \verb|INT(11)|, \verb|NOT NULL|,
			\item \textbf{pytania} $\to$  \verb|text COLLATE cp1250_polish_ci|, \verb|NOT NULL|,
			\item \textbf{odpowiedzi} $\to$  \verb|text COLLATE cp1250_polish_ci|, \verb|NOT NULL|,
			\item \textbf{poprawne} $\to$  \verb|text COLLATE cp1250_polish_ci|, \verb|NOT NULL|,
			\item \textbf{czas} $\to$  \verb|INT(11)|, \verb|NOT NULL| 
			\item \textbf{ocena} $\to$ typu - \verb|INT(11)|, \verb|NOT NULL| 
		\end{itemize}
		
				\vspace{0.5cm}
				Tabela \textbf{pytania}:
				\begin{itemize}
					\item \textbf{id} $\to$ typu - \verb|INT(11)|,\verb|NOT NULL|, \verb|AUTO_INCREMENT|,
					\item \textbf{idDzialu} $\to$ typu - \verb|INT(11)|, \verb|NOT NULL| 
					\item \textbf{idKolokwium} $\to$ typu - \verb|INT(11)|, \verb|NOT NULL|,
					\item \textbf{pytanie} $\to$  \verb|text COLLATE cp1250_polish_ci|, \verb|NOT NULL|,
					\item \textbf{odp1} $\to$  \verb|text COLLATE cp1250_polish_ci|, \verb|NOT NULL|,
					\item \textbf{odp2} $\to$  \verb|text COLLATE cp1250_polish_ci|, \verb|NOT NULL|,
					\item \textbf{odp3} $\to$  \verb|text COLLATE cp1250_polish_ci|, \verb|NOT NULL|,
					\item \textbf{odp4} $\to$  \verb|text COLLATE cp1250_polish_ci|, \verb|NOT NULL|,
					\item \textbf{poprawne} $\to$  \verb|INT(11)|, \verb|NOT NULL|,
				\end{itemize}
				
				\vspace{0.5cm}
				Tabela \textbf{kolokwia}:
				\begin{itemize}
					\item \textbf{id} $\to$ typu - \verb|INT(11)|,\verb|unsigned NOT NULL|, \verb|AUTO_INCREMENT|,
					\item \textbf{idDzialy} $\to$ typu - \verb|INT(11)|, \verb|NOT NULL|,
					\item \textbf{name} $\to$ typu - \verb|VARCHAR(30)|,\verb|text COLLATE cp1250_polish_ci|, \verb|NOT NULL| 
					\end{itemize}				
					
				\vspace{0.5cm}
				Tabela \textbf{kolokwia}:
				\begin{itemize}
					\item \textbf{id} $\to$ typu - \verb|INT(11)|,\verb|unsigned NOT NULL|, \verb|AUTO_INCREMENT|,
					\item \textbf{name} $\to$ typu - \verb|VARCHAR(30)|,\verb|text COLLATE cp1250_polish_ci|, \verb|NOT NULL| 
					\end{itemize}									
		
		 
\subsection{Sprint Review/Demo}
Sprint został zakończony powodzeniem, wszystkie główne założone zadania zostały zrealizowane w ustalonym czasie.  Demo zostało zaprezentowane 13 grudnia. Czas przeznaczony na wykonanie to 3 tygodnie. W ramach sprintu utworzyliśmy bazę danych, która zawiera kilku dodanych "na sztywno" użytkowników(z możliwością dodawania kolejnych - formularz rejestracji z poziomu egzaminatora nie jest jeszcze dostępny), dzięki czemu możliwe jest zalogowanie się. Logowanie odbywa w utworzonym w poprzednim sprincie formularzu, który do tej pory pełnił jedynie funkcję poglądową. Została stworzona weryfikacja danych wprowadzanych podczas logowania, która weryfikuje czy w bazie danych, w tabeli "users" znajduje się rekord zawierający wartości 'user' oraz 'password' zgodne z tymi wprowadzonymi w formularzu. W przypadku kiedy użytkownik  poda poprawne dane nastąpi przejście na stronę główną egzaminatora. Wyświetlane na niej są imie i nazwisko zalogowanej osoby. Użytkownik ma możliwość wylogowania się po którym następuje powrót do wcześniej wspomnianego formularza. Na chwile obecną po wpisaniu błędnego loginu lub hasła nie wyświetla się żaden komunikat(zostanie to dodane w kolejnym sprincie)

\section{Sprint 3}

\subsection{Cel}  Lista egzaminów oraz panel konta użytkownika.

\subsection{Sprint Planning/Backlog}

\paragraph{Tytuł zadania.} Lista egzaminów
\begin{itemize}
\item Estymata: XL
\end{itemize} 

\paragraph{Tytuł zadania.} Panel konta użytkownika
\begin{itemize}
\item Estymata: L
\end{itemize} 


\subsection{Realizacja}

\paragraph{Tytuł zadania.}  Lista egzaminów
\subparagraph{Wykonawca.} Kamil Orzechowski, Przemywsław Rysiewicz.
\subparagraph{Realizacja.} Po zalogowaniu, w sekcji MENU, znajduje się lista egzaminów możliwych do rozpoczęcia. Po wybraniu jednego z listy możliwe jest rozpoczęcie egzaminu. Sekcja została podzielona na 2 kategorie: 'dzialy' oraz 'kolokwia'. Ma to na celu umożliwić grupowanie tematycznie egzaminów w finalnej wersji Egzaminatora. Menu wykonane z użyciem JavaScript działa jako lista rozwijana, po kliknięciu na interesujący nas dział wyświetlają się dostępne dla nas kolokwia. W tym zadaniu zostało dodane działające już w pełni wypełnianie egzaminu. Po wybraniu konkretnego egzaminu wyświetla się przycisk wraz z nazwą wybranego kolokwium umożliwiający rozpoczęcie. Po kliknięciu na przycisk zostajemy przeniesieni do strony egzaminowania.  Został utworzony panel zawierający zegar odliczający do końca oraz panel wyboru pytania. Do każdego pytania zostały dodane 4 odpowiedzi z czego 1 jest poprawna. Dodaliśmy 2 możliwości zmiany pytania: jedna umożliwia przejście o jedno dalej, druga natomiast pozwala przechodzenie z dowolnego do dowolnego. Dodaliśmy również licznik zdobytych punktów wyświetlający się po zakończeniu (zakończenie może nastąpić po upływie czasu lub po wciśnięciu przycisku). Informacje danej sesji przechowywane są w bazie danych w tabeli 'session'. Jako, iż jest to główne założenie, najważniejsze zadanie naszego całego projektu poświęciliśmy mu najwięcej czasu. 

\paragraph{Tytuł zadania.} Panel konta użytkownika.
\subparagraph{Wykonawca.} Paweł Tomasiak.
\subparagraph{Realizacja.} Został utworzony panel użytkownika zawierający wszystkie funkcje do jakich dostęp ma mieć użytkownik nie będący administratorem, . Przez "Panel konta użytkownika" w naszym projekcie rozumiemy wszystko do czego uzyskujemy dostęp po zalogowaniu min. wspomniana wcześniej Lista Egzaminów 
 
\subsection{Sprint Review/Demo}
Udało nam się spełnić główne założenia, sprint został przyjęty. Demo odbyło się 10 stycznia. Podczas trwania prezentacji doszliśmy do wniosków, iż warto usprawnić min. Listę Egzaminów o kilka opcjonalnych elementów, takich jak zablokowanie możliwości wielokrotnego uzupełniania jednego egzaminu oraz sposobu  ich wyświetlania w menu (na chwile obecną wyswietlają się wszystkie jednak powinny się wyświetlać te, które muszę w danej chwili uzupełnić) .  Został dodany komunikat informujący o błędnym loginie lub haśle podanym podczas logowania, który miał się znaleźć w poprzednim sprincie jednak został wtedy pominięty. 

\section{Sprint 4}

\subsection{Cel} Przydzielanie egzaminów, zarządanie studentami, rejestracja. 

\subsection{Sprint Planning/Backlog}

\paragraph{Tytuł zadania.} Przydzielanie egzaminów.
\begin{itemize}
\item Estymata: M
\end{itemize} 

\paragraph{Tytuł zadania.} Zarządanie studentami.
\begin{itemize}
\item Estymata: M
\end{itemize} 

\paragraph{Tytuł zadania.} Rejestracja.
\begin{itemize}
\item Estymata: L
\end{itemize} 

\paragraph{Tutaj dodawać kolejne zadania}

\subsection{Realizacja}

\paragraph{Tytuł zadania.} Przydzielanie egzaminów.
\subparagraph{Wykonawca.} Kamil Orzechowski
\subparagraph{Realizacja.} Została utworzona nowa podstrona dostępna tylko dla administratora (wszystkie dzisiejsze zadania oparte są na tej podstronie) Przez przydzielanie studentów rozumiemy możliwość dodawania nowych pytań do wybranego działu, z możliwością dodania 4 pytań i oznaczenia jednej poprawnej. Zadanie to okazało się najtrudniejsze 

\paragraph{Tytuł zadania.} Zarządanie studentami.
\subparagraph{Wykonawca.} Przemek Rysiewicz
\subparagraph{Realizacja.} Ponownie jest to zakładka, która zgodnie z naszym zamiarem dostępna jest tylko dla administratora. Pozwala na wyświetlenie wyników z egzaminów uzupełnionych przez studentów. Wyświetlane informacje są w postaci tabeli, która zawiera dane studenta, ocenę oraz datę wykonania testu. Z racji, iż jest to głównie wyświetlanie informacji zawartych w tabeli w bazie danych zadanie to nie sprawiło nam większych problemów, wykonanie zajęło około 2 dni po 1-2h pracy dziennie.

\paragraph{Tytuł zadania.} Rejestracja.
\subparagraph{Wykonawca.} Przemysław Rysiewicz, Kamil Orzechowski
\subparagraph{Realizacja.} To zadanie podobnie jak dwa poprzednie zrealizowane jest na bazie podstrony dostępnej dla administratora (admin.php). Zgodznie z naszym założeniem tworzeniem nowych kont użytkowników zajmuje się administrator, zwykły użytkownik nie ma możliwości rejestracji. Do tego wykonania tego zadania przystąpiły dwie osoby co przyczyniło się do szybkiego postępu pracy.  Wykonanie zadania zajęło 3 dni po ok. 2h pracy dziennie.

\subsection{Sprint Review/Demo}
Ostatnia zaplanowana prezentacja postępów odbyła się 24 stycznia. Zaprezentowany został panel administratora, w którym znalazły się: Rejestracja, dodawanie studentów oraz pytań. Z racji iż ten sprint nie był wykonywany przez głównego developera lecz w całości przez dwóch pobocznych deweloperów nie był on w pełni satysfakcjonujący jednak spełniał główne założenia. Tworzenie projektu w dniu dzisiejszym zostało zakończone. Projekt oddany oraz oceniony na oceny 4.5 oraz 5.0 dla głównego developera.



\begin{thebibliography}{9}

\bibitem{Cov} P.Tomasiak, {\em Wiedza z jego głowy}, Nowy Sącz, 2017-2018 :).

\end{thebibliography}

\end{document}


\textbf{}
