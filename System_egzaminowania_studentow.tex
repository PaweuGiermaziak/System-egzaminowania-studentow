\documentclass[a4paper]{article}
\usepackage{polski}
\usepackage[cp1250]{inputenc}
\usepackage{url}

\title{\bf{System egzaminowania studentów}}
\author{{\em Paweł Tomasiak, Kamil Orzechowski, Przemek Rysiewicz}}
\date{}

\begin{document}

\begin{titlepage}
\maketitle
\thispagestyle{empty}
\bigskip
\begin{center}
Zespołowe przedsięwzięcie inżynierskie\\[2mm]

Informatyka\\[2mm]

Rok. akad. 2017/2018, sem. I\\[2mm]

Prowadzący: dr hab. Marcin Mazur
\end{center}
\end{titlepage}

\tableofcontents
\thispagestyle{empty}

\newpage

\section{Opis projektu}

\subsection{Członkowie zespołu}

\begin{enumerate}
\item Paweł Tomasiak (kierownik projektu)
\item Przemysław Rysiewicz
\item Kamil Orzechowski
\end{enumerate}

\subsection{Cel projektu (produkt)}

Celem projektu jest stworzenie platformy ułatwiającej osobie prowadzącej zadania dydaktyczne egzaminowanie studentów. Platforma ma w założeniu być substytutem testu wiedzy z pełną automatyzacją obsługi oceniania.

\subsection{Potencjalny odbiorca produktu (klient)}

Odbiorcą produktu jest osoba prowadzaca zadania dydaktyczne wymagajace oceniania wiedzy osób nauczanch. Do tej grupy zalicza sie nauczycieli i wykładowców.

\subsection{Metodyka}

Projekt będzie realizowany przy użyciu (zaadaptowanej do istniejących warunków) metodyki {\em Scrum}. 

\section{Wymagania użytkownika}
\subsection{User story 1}

Jako użytkownik chcę aby interfejs graficzny był przejrzysty tak abym mógł intuicyjnie poruszać się po stronie

\subsection{User story 2}

Jako użytkownik chcę mieć łatwy dostęp do interesującego mnie kolokwium żeby od razu móc przystąpić do testu

\subsection{User story 3}

Jako użytkownik chcę widzieć wynik jaki osiągnąłem podczas egzaminu aby móc śledzić swoje postępy

\subsection{User story 4}

Jako użytkownik chcę mieć możliwość logowania sie na istniejące konta w bazie, żeby w każdej chwili mieć dostęp do testów

\subsection{User story 5}

Jako użytkownik chcę mieć możliwość wybrania konkretnego zespołu (roku akademickiego), żeby widzieć tylko dotyczące mnie egzaminy

\subsection{User story 6}

Jako administrator chcę aby skrypt przesłał dane o rezultacie testu na serwer tak abym mógł obejrzeć wyniki każdego użytkownika

\subsection{User story 7}

Jako administrator chcę aby skrypt generował pyrania z puli, po to żeby umożliwić dobór różnych pytań dla każdego studenta

\subsection{User story 8}

Jako administrator chcę aby wybór poprawnej odpowiedzi mógł odbywać się poprzez zaznaczenie więcej niż jednej z kilku możliwych, abym miał więcej możliwości w tworzeniu pytań

\subsection{User story 9}

Jako administrator chcę żeby ilość czasu na wypenienie testu była ograniczona, żeby egzamin nie trwał dłużej jest niż to konieczne

\subsection{User story 10}

Jako administrator chcę aby po zakończeniu dostępnego czasu skrypt wyświetlał uzyskany wynik, żeby każdy dysponował taką samą ilością czasu

\subsection{User story 11}

Jako administrator chcę aby skrypt zapisywał wynik otrzymany przez studenta na serwerze, tak abym miał do niego później dostęp

\subsection{User story 12}

Jako administrator wymagam dostepu do danych przechowywanych na serwerze bazy w sposób przystępny i zorganizowany, tak abym mógł łatwo przeglądać poszczególne wyniki

\subsection{User story 13}

Jako administrator chcę mieć możliwość dodawania pytań do bazy w celu swobodnego dostosowywania zadań egzaminu do aktualnych potrzeb

\subsection{User story 14}

Jako wykładowca wymagam możliwości dodawania kont sudentów, tak abym mógł bezpiecznie zarządzać ich ilością

\section{Harmonogram}

\subsection{Rejestr zadań (Product Backlog)}

\begin{itemize}
\item Data rozpoczęcia: 2017-10-18.
\item  Data zakończenia: 2017-11-15.
\end{itemize}


\subsection{Sprint 1}

\begin{itemize}
\item Data rozpoczęcia: 2017-11-15.
\item  Data zakończenia: 2017-11-22.
\item Scrum Master: Kamil Orzechowski
\item Product Owner: Przemysław Rysiewicz
\item Development Team: Paweł Tomasiak, Przemysław Rysiewicz, Kamil Orzechowski.
\end{itemize}

\subsection{Sprint 2}

\begin{itemize}
\item Data rozpoczęcia: 2017-11-22.
\item  Data zakończenia: 2017-12-13.
\item Scrum Master: Kamil Orzechowski
\item Product Owner: Przemysław Rysiewicz
\item Development Team: Paweł Tomasiak, Przemysław Rysiewicz, Kamil Orzechowski.
\end{itemize}

\subsection{Sprint 3}

\begin{itemize}
\item Data rozpoczęcia: 2017-12-13.
\item  Data zakończenia: 2017-12-27.
\item Scrum Master: Paweł Tomasiak
\item Product Owner: Kamil Orzechowski
\item Development Team:
\end{itemize}

\subsection{Sprint 4}

\begin{itemize}
\item Data rozpoczęcia: 2017-12-27.
\item  Data zakończenia: 2018-01-17.
\item Scrum Master: Przemysław Rysiewicz
\item Product Owner: Paweł Tomasiak
\item Development Team:
\end{itemize}


\section{Product Backlog}

\subsection{Backlog Item 1}
\paragraph{Tytuł zadania.} Prototyp egzaminatora.
\paragraph{Opis zadania.} Stworzenie działającego prototypu strony będącego w stanie egzaminować uczniów.
\paragraph{Priorytet.} Wysoki.
\paragraph{Definition of Done.} Prototyp zawiera podgląd strony, która posłuży do przeprowadzania egzaminu, podstrony: logowanie, lista egzaminów, zakończenie egzaminu oraz administrator.

\subsection{Backlog Item 2}
\paragraph{Tytuł zadania.} Baza danych.
\paragraph{Opis zadania.} Stworzenie struktur bazodanowych w środowisku MySQL.
\paragraph{Priorytet.} Wysoki.
\paragraph{Definition of Done.} Stworzenie bazy danych posiadającej w tabelach wartości takie jak: id, nazwa użytkownika, hasło, email, oceny.

\subsection{Backlog Item 3}
\paragraph{Tytuł zadania.} Rejestracja.
\paragraph{Opis zadania.} Możliwość utowrzenia nowego konta.
\paragraph{Priorytet.} Średni.
\paragraph{Definition of Done.} Stworzenie formularza umożliwiającego każdemu założenie nowego konta. Pomyślna rejestracja utworzy nowy rekord w bazie danych.

\subsection{Backlog Item 4}
\paragraph{Tytuł zadania.} Weryfikacja.
\paragraph{Opis zadania.} Zweryfikowanie wprowadzonych danych podczas rejestracji.
\paragraph{Priorytet.} Średni.
\paragraph{Definition of Done.} Sprawdzenie, czy podane zostały wszystkie wymagane dane, czy posiadają odpowiedni format oraz czy liczba wprowadzonych znaków mieści się w ustalonym przedziale. Jeśli nie został spełniony któryś z warunków zostanie wyświetlony komunikat.

\subsection{Backlog Item 5}
\paragraph{Tytuł zadania.} Logowanie.
\paragraph{Opis zadania.} Umożliwienie zalogowania na istniejące konto.
\paragraph{Priorytet.} Wysoki.
\paragraph{Definition of Done.} Formularz logowania umożliwiający rozpoczęcie nowej sesji oraz dostęp do przeznaczonych dla niego egzaminów. Zalogowanie nastąpi jeśli wprowadzone w formularzu dane (login,hasło) znajdują się już w intniejącej bazie.

\subsection{Backlog Item 6}
\paragraph{Tytuł zadania.} Lista egzaminów.
\paragraph{Opis zadania.} Panel boczny zawierający listę egzaminów.
\paragraph{Priorytet.} Wysoki.
\paragraph{Definition of Done.} Panel znajdujący się po lewej stronie zawierający rozwijaną listę aktualnie dostępnych egzaminów.

\subsection{Backlog Item 7}
\paragraph{Tytuł zadania.} Panel konta użytkownika.
\paragraph{Opis zadania.} Przegląd danych konta użytkownika oraz wyniki egzaminów .
\paragraph{Priorytet.} Śrerdni.
\paragraph{Definition of Done.} Użytkownik ma możliwość przeglądania danych wprowadzonych podczas rejestracji oraz zmiany niektórych z nich(adres email, hasło) , może przeglądać swoje wyniki z egzaminów .

\subsection{Backlog Item 8}
\paragraph{Tytuł zadania.} Przydzielanie egzaminów.
\paragraph{Opis zadania.} Dobranie egzaminów przez administratora.
\paragraph{Priorytet.} Niski
\paragraph{Definition of Done.} Użytkownik z prawami administratora ma możliwość odblokowywania oraz blokowania dostępu do konkretnych egzaminów zalogowanemu użytkownikowi, ma dostęp także do wyników osiągniętych przez studenta.

\subsection{Backlog Item 9}
\paragraph{Tytuł zadania.} Zarządanie studentami.
\paragraph{Opis zadania.} Przydzielanie studentów do wybranego roku, usuwanie.
\paragraph{Priorytet.} Niski
\paragraph{Definition of Done.} Użytkownik z prawami administratora ma możliwość przydzielenia studenta do wybranego roku, ma też możliwość usunięcia konta użytkownika .

\subsection{Backlog Item 10}
\paragraph{Tytuł zadania.} Serwer
\paragraph{Opis zadania.} Utworzenie serwera, na którym działał będzie egzaminator
\paragraph{Priorytet.} Wysoki
\paragraph{Definition of Done.} Skrypt przełącza użytkownika na stronę wyboru egzaminów po wpisaniu poprawnych danych logowania. 

\subsection*{Tutaj dodawać kolejne zadania}

\section{Sprint 1} Prototyp egzaminatora
\subsection{Cel} Przedstawienie projektu graficznego w wersji HTML
\subsection{Sprint Planning/Backlog}

\paragraph{Tytuł zadania.} Prototyp egzaminatora
\begin{itemize}
\item Estymata: M
\end{itemize}

\subsection{Realizacja}

\paragraph{Tytuł zadania.} Prototyp egzaminatora
\subparagraph{Wykonawca.} Paweł Tomasiak
\subparagraph{Realizacja.} Została utworzona strona logowania, która wymaga podania loginu i hasła; strona zawierająca rozwijaną listę egzaminów, z możliwością rozpoczęcia wybranego; stronę przedstawiającą wygląd rozpoczętego egzaminu, zawierająca licznik czasowy, panel wyboru pytań i okno z wybranym pytaniem wraz z odpowiedziami; strona wyświetlana po zakończeniu egzaminu zawierająca wynik; stron zawierająca panel administratora przedstawiający wyniki studentów.

\subsection{Sprint Review/Demo}
Założony cel sprintu został spełniony. Wszystkie założenia zostały osiągnięte w ustalonym czasie. Zrealizowany został prototyp zawierający strony:
\begin{itemize}
\item strona logowania
\item lista egzaminów
\item podgląd rozpoczętego egzaminu
\item wynik egzaminu 
\item panel administratora
\end{itemize}

\section{Sprint 2}

\subsection{Cel} Serwer, logowanie, weryfikacja, baza danych

\subsection{Sprint Planning/Backlog}

\paragraph{Tytuł zadania.} Serwer
\begin{itemize}
\item Estymata: L
\end{itemize}

\paragraph{Tytuł zadania.} Logowanie
\begin{itemize}
\item Estymata: S
\end{itemize}

\paragraph{Tytuł zadania.} Weryfikacja
\begin{itemize}
\item Estymata: M
\end{itemize}

\paragraph{Tytuł zadania.} Baza danych
\begin{itemize}
\item Estymata: L
\end{itemize}

\subsection{Realizacja}

\paragraph{Tytuł zadania.} Serwer
\subparagraph{Wykonawca.} Kamil Orzechowski
\subparagraph{Realizacja.} Sprawozdanie z realizacji zadania (w tym ocena zgodności z estymatą). Kod programu (środowisko \texttt{verbatim}): \begin{verbatim}
for (i=1; i<10; i++)
...
\end{verbatim}.

\paragraph{Tytuł zadania.} Logowanie
\subparagraph{Wykonawca.} Paweł Tomasiak
\subparagraph{Realizacja.} Sprawozdanie z realizacji zadania (w tym ocena zgodności z estymatą). Kod programu (środowisko \texttt{verbatim}): \begin{verbatim}
for (i=1; i<10; i++)
...
\end{verbatim}.

\paragraph{Tytuł zadania.} Weryfikacja
\subparagraph{Wykonawca.} Paweł Tomasiak
\subparagraph{Realizacja.} Sprawozdanie z realizacji zadania (w tym ocena zgodności z estymatą). Kod programu (środowisko \texttt{verbatim}): \begin{verbatim}
for (i=1; i<10; i++)
...
\end{verbatim}.


\paragraph{Tytuł zadania.} Baza Danych
\subparagraph{Wykonawca.} Przemysław Rysiewicz
\subparagraph{Realizacja.} Sprawozdanie z realizacji zadania (w tym ocena zgodności z estymatą). Kod programu (środowisko \texttt{verbatim}): \begin{verbatim}
for (i=1; i<10; i++)
...
\end{verbatim}.


\subsection{Sprint Review/Demo}
Sprawozdanie z przeglądu Sprint'u -- czy założony cel (przyrost) został osiągnięty oraz czy wszystkie zaplanowane Backlog Item'y zostały zrealizowane? Demostracja przyrostu produktu.

\section{Sprint 3}

\subsection{Cel}  Lista egzaminów, panel konta użytkownika

\subsection{Sprint Planning/Backlog}

\paragraph{Tytuł zadania.} Tytuł.
\begin{itemize}
\item Estymata: szacowana czasochłonność (w ,,koszulkach'').
\end{itemize}

\paragraph{Tytuł zadania.} Tytuł.
\begin{itemize}
\item Estymata: szacowana czasochłonność (w ,,koszulkach'').
\end{itemize}

\paragraph{Tutaj dodawać kolejne zadania}

\subsection{Realizacja}

\paragraph{Tytuł zadania.} Tytuł.
\subparagraph{Wykonawca.} Wykonawca.
\subparagraph{Realizacja.} Sprawozdanie z realizacji zadania (w tym ocena zgodności z estymatą). Kod programu (środowisko \texttt{verbatim}): \begin{verbatim}
for (i=1; i<10; i++)
...
\end{verbatim}.

\paragraph{Tytuł zadania.} Tytuł.
\subparagraph{Wykonawca.} Wykonawca.
\subparagraph{Realizacja.} Sprawozdanie z realizacji zadania (w tym ocena zgodności z estymatą). Kod programu (środowisko \texttt{verbatim}): \begin{verbatim}
for (i=1; i<10; i++)
...
\end{verbatim}.

\paragraph{Tutaj dodawać kolejne zadania}


\subsection{Sprint Review/Demo}
Sprawozdanie z przeglądu Sprint'u -- czy założony cel (przyrost) został osiągnięty oraz czy wszystkie zaplanowane Backlog Item'y zostały zrealizowane? Demostracja przyrostu produktu.

\section{Sprint 4}

\subsection{Cel} Przydzielanie egzaminów, zarządanie studentami.

\subsection{Sprint Planning/Backlog}

\paragraph{Tytuł zadania.} Tytuł.
\begin{itemize}
\item Estymata: szacowana czasochłonność (w ,,koszulkach'').
\end{itemize}

\paragraph{Tytuł zadania.} Tytuł.
\begin{itemize}
\item Estymata: szacowana czasochłonność (w ,,koszulkach'').
\end{itemize}

\paragraph{Tutaj dodawać kolejne zadania}

\subsection{Realizacja}

\paragraph{Tytuł zadania.} Tytuł.
\subparagraph{Wykonawca.} Wykonawca.
\subparagraph{Realizacja.} Sprawozdanie z realizacji zadania (w tym ocena zgodności z estymatą). Kod programu (środowisko \texttt{verbatim}): \begin{verbatim}
for (i=1; i<10; i++)
...
\end{verbatim}.

\paragraph{Tytuł zadania.} Tytuł.
\subparagraph{Wykonawca.} Wykonawca.
\subparagraph{Realizacja.} Sprawozdanie z realizacji zadania (w tym ocena zgodności z estymatą). Kod programu (środowisko \texttt{verbatim}): \begin{verbatim}
for (i=1; i<10; i++)
...
\end{verbatim}.

\paragraph{Tutaj dodawać kolejne zadania}


\subsection{Sprint Review/Demo}
Sprawozdanie z przeglądu Sprint'u -- czy założony cel (przyrost) został osiągnięty oraz czy wszystkie zaplanowane Backlog Item'y zostały zrealizowane? Demostracja przyrostu produktu.


\section*{Tutaj dodawać kolejne Sprint'y}


\begin{thebibliography}{9}

\bibitem{Cov} S. R. Covey, {\em 7 nawyków skutecznego działania}, Rebis, Poznań, 2007.

\bibitem{Oet} Tobias Oetiker i wsp., Nie za krótkie wprowadzenie do systemu \LaTeX  \ $2_\varepsilon$, \url{ftp://ftp.gust.org.pl/TeX/info/lshort/polish/lshort2e.pdf}

\bibitem{SchSut} K. Schwaber, J. Sutherland, {\em Scrum Guide}, \url{http://www.scrumguides.org/}, 2016.

\bibitem{apr} \url{https://agilepainrelief.com/notesfromatooluser/tag/scrum-by-example}

\bibitem{us} \url{https://www.tutorialspoint.com/scrum/scrum_user_stories.htm}

\end{thebibliography}

\end{document}
